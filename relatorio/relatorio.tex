\documentclass[brazil]{article}


% ================================== PACOTES ===================================

\usepackage[brazil]{babel}
\usepackage{textcomp} % ajuda hifenação
\usepackage[utf8]{inputenc}
\usepackage[pdftex]{hyperref}

\usepackage{fullpage}
\usepackage{setspace}
\usepackage{url}

\usepackage{enumerate}

% incrementando o espaçamento entre parágrafos (o padrão é 0, estamos
% adicionando uma linha vazia)
\addtolength{\parskip}{\baselineskip}

% Inicio -----------------------------------------------------------------------

\begin{document}
\title{MAC0463/5743 Computação Móvel\\\medskip
    Relatório -- EP1}
\author{Samuel Plaça de Paula
        \and Thiago de Gouveia Nunes}
\date{Universidade de São Paulo, abril de 2013}

\maketitle
\pagestyle{plain}
\footskip=25pt

% Resumo -----------------------------------------------------------------------

\begin{abstract}
    Neste documento damos uma visão geral sobre a aplicação criada para o
    primeiro exercício-programa da disciplina de Computação Móvel do IME-USP
    no primeiro semestre de 2013\footnotemark[1].
    Trata-se de um visualizador de \emph{feeds}
    RSS de notícias da USP -- no caso, o serviço de divulgação por nós
    escolhido foi o USP Eventos\footnotemark[2].
    Discutimos aqui brevemente a justificativa para a criação da aplicação,
    as características do sistema produzido e sua implementação.
\end{abstract}

\footnotetext[1]{Mais informações em
\url{http://grenoble.ime.usp.br/~gold/cursos/2013/movel/}}

\footnotetext[2]{Acessível em \url{http://www.eventos.usp.br/}}

%------------------------------------------------------------------------------

\section{Motivação}
% Apresentar brevemente as razões que os levaram a implementar o app.
% Descreva a situação-problema a qual o app está relacionado bem como a
% justificativa para sua criação.
Estar na USP significa ter ao alcance coisas como uma ótima formação,
excelentes bibliotecas, contato com professores e colegas competentes, etc.
Mas significa também estar inserido em um ecossistema que oferece esportes,
concertos, conferências, palestras, enfim, muito mais do que abrangem os
elementos formais de um curso.

É comum encontrar alunos que não desfrutam nem de uma fração disso, geralmente
por falta de conhecimento -- às vezes em decorrência de preguiça de buscar
informações. Essas pessoas apenas vão para a faculdade, assistem a suas aulas,
usam computadores e biblioteca para realização de estudo e trabalhos, e voltam
para casa. Trata-se de algo bastante negativo, visto que a universidade oferece
tanto mais, e que é desejável para todos os envolvidos que mais pessoas
participem do diálogo promovido pelos eventos realizados na USP.

O USP Eventos serve como
meio de divulgação dos eventos que acontecem na universidade, como palestras,
concertos, sessões de cinema, debates, etc. No entanto, ao
menos vendo o exemplo de nossos colegas, não parece que uma parcela
significativa da comunidade universitária usa esse serviço.
Desenvolver um aplicativo que se proponha a oferecer tais informações de
maneira mais cômoda e próxima do usuário, através do uso de dispositivos
móveis, é uma tentativa de colaborar para que muito mais pessoas passem a
acompanhar esse rico aspecto do dia-a-dia da universidade.

%------------------------------------------------------------------------------

\section{Descrição}
% Apresentar as funcionalidades do App sob o ponto de vista do usuário.
% A descrição deve ser orientada pelas telas do app e seu fluxo. O estilo
% "Manual de Usuário" é adequado a esta seção, apresentando as funcionalidades
% de cada tela e os passos necessários ao uso de cada funcionalidade
% disponibilizada.

%------------------------------------------------------------------------------

\section{Implementação}
% Apresentar a estrutura geral da implementação do app e partes significativas
% do funcionamento do código, caso necessário, evidenciando o trabalho de
% vocês sob o ponto de vista do desenvolvedor. Aqui um diagrama facilita
% bastante a compreensão (mesmo que informal).

%------------------------------------------------------------------------------

\section{Conclusões}
% Apresentar sucintamente suas constatações sobre o processo de aprendizado e
% desenvolvimento do app.

\end{document}
