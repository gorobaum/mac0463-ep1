\documentclass[brazil]{article}


% ================================== PACOTES ===================================

\usepackage[brazil]{babel}
\usepackage{textcomp} % ajuda hifenação
\usepackage[utf8]{inputenc}
\usepackage[pdftex]{hyperref}

\usepackage{fullpage}
\usepackage{setspace}
\usepackage{url}

\usepackage{enumerate}

% incrementando o espaçamento entre parágrafos (o padrão é 0, estamos
% adicionando uma linha vazia)
\addtolength{\parskip}{\baselineskip}

% Inicio -----------------------------------------------------------------------

\begin{document}
\title{MAC0463/5743 Computação Móvel\\\medskip
    Relatório -- EP1}
\author{Samuel Plaça de Paula
        \and Thiago de Gouveia Nunes}
\date{Universidade de São Paulo, abril de 2013}

\maketitle
\pagestyle{plain}
\footskip=25pt

% Resumo -----------------------------------------------------------------------

\begin{abstract}
    Neste documento damos uma visão geral sobre a aplicação criada para o
    primeiro exercício-programa da disciplina de Computação Móvel do IME-USP
    no primeiro semestre de 2013\footnotemark[1].
    Trata-se de um visualizador de \emph{feeds}
    RSS de notícias da USP -- no caso, o serviço de divulgação por nós
    escolhido foi o USP Eventos\footnotemark[2].
    Discutimos aqui brevemente a justificativa para a criação da aplicação,
    as características do sistema produzido e sua implementação.
\end{abstract}

\footnotetext[1]{Mais informações em
\url{http://grenoble.ime.usp.br/~gold/cursos/2013/movel/}}

\footnotetext[2]{Acessível em \url{http://www.eventos.usp.br/}}

%------------------------------------------------------------------------------

\section{Motivação}
% Apresentar brevemente as razões que os levaram a implementar o app.
% Descreva a situação-problema a qual o app está relacionado bem como a
% justificativa para sua criação.
Estar na USP significa ter ao alcance coisas como uma ótima formação,
excelentes bibliotecas, contato com professores e colegas competentes, etc.
Mas significa também estar inserido em um ecossistema que oferece esportes,
concertos, conferências, palestras, enfim, muito mais do que abrangem os
elementos formais de um curso.

É comum encontrar alunos que não desfrutam nem de uma fração disso, geralmente
por falta de conhecimento -- às vezes em decorrência de preguiça de buscar
informações. Essas pessoas apenas vão para a faculdade, assistem a suas aulas,
usam computadores e biblioteca para realização de estudo e trabalhos, e voltam
para casa. Trata-se de algo bastante negativo, visto que a universidade oferece
tanto mais, e que é desejável para todos os envolvidos que mais pessoas
participem do diálogo promovido pelos eventos realizados na USP.

O USP Eventos serve como
meio de divulgação dos eventos que acontecem na universidade, como palestras,
concertos, sessões de cinema, debates, etc. No entanto, ao
menos vendo o exemplo de nossos colegas, não parece que uma parcela
significativa da comunidade universitária usa esse serviço.
Desenvolver um aplicativo que se proponha a oferecer tais informações de
maneira mais cômoda e próxima do usuário, através do uso de dispositivos
móveis, é uma tentativa de colaborar para que muito mais pessoas passem a
acompanhar esse rico aspecto do dia-a-dia da universidade.

%------------------------------------------------------------------------------

\section{Descrição}
% Apresentar as funcionalidades do App sob o ponto de vista do usuário.
% A descrição deve ser orientada pelas telas do app e seu fluxo. O estilo
% "Manual de Usuário" é adequado a esta seção, apresentando as funcionalidades
% de cada tela e os passos necessários ao uso de cada funcionalidade
% disponibilizada.
O app tem um visual bastante direto. Quase todo o corpo se destina à exibição
dos feeds recuperados. Os feeds são separados de acordo com suas categorias
e agrupados sob listas colapsadas que podem ser estendidas ao tocar-se
na barra contendo o nome da categoria. Para cada feed, são exibidos seu título
e primeiros caracteres da notícia. Toda a linha é um link para a notícia
original.

Na parte superior, há um botão à esquerda e outro à direita.

O botão da esquerda é o ``+ Feeds'', que se destina a carregar um número maior
de feeds (o padrão é 10). Os feeds assim carregados podem ser
tanto mais antigos que os já presentes no sistema, quanto mais novos, isto é,
feeds que o USP Eventos subiu desde a última verificação feita pelo
app.

O botão à direita é o ``Configurações'', que abre um menu em que se podem
selecionar os feeds que o usuário deseja ver. Ao marcar as categorias
desejadas e tocar ``salvar configurações'', o programa recarrega os feeds
apenas das categorias desejadas, buscando novos feeds dessas categorias se
possível.

%------------------------------------------------------------------------------

\section{Implementação}
% Apresentar a estrutura geral da implementação do app e partes significativas
% do funcionamento do código, caso necessário, evidenciando o trabalho de
% vocês sob o ponto de vista do desenvolvedor. Aqui um diagrama facilita
% bastante a compreensão (mesmo que informal).
Para controlar o comportamento das páginas, usamos o framework jQuery Mobile.
Usamos Web SQL para criar um BD com os dados dos feeds recuperados.
Para ler os feeds, usamos o Google JSAPI.

O app desenvolvido consiste principalmente dos módulos \texttt{feeder.js} e
\texttt{storage.js}.

O primeiro é o que define o comportamento do objeto \texttt{feeder}, que
recupera e exibe os feeds. Ele inicializa as variáveis necessárias e
mantém o app funcionando via uma chamada a
\texttt{setInterval}, que executa a função \texttt{loadFeeds}, a ser descrita
adiante, a cada 30
segundos. Além disso, sempre que uma nova configuração de tipos de feeds
desejados é salva, também é chamada \texttt{loadFeeds}.

Já em \texttt{storage.js} define-se uma classe \texttt{Storage}, que é um
gerenciador de BD. O \texttt{feeder} possui uma instância de \texttt{Storage}
(chamada \texttt{store}), que usa para criar e manter o BD com os dados dos
feeds.

A função de \texttt{loadFeeds} é atualizar a visualização de feeds com as
configurações atuais de categorias desejadas. Ela chama uma função do
\texttt{Storage} que obtém as categorias de feeds presentes no BD e termina
chamando uma função (anônima, passada como o parâmetro \texttt{callback}) que
para cada categoria verifica se esta está sendo requisitada pelo usuário
(isto é, marcada na checklist). Em caso afirmativo, mostra os feeds dessa
categoria já armazenados e tenta obter novos feeds dessa categoria; em caso
negativo, esconde os feeds dessa categoria que estão armazenados.

%------------------------------------------------------------------------------

\section{Conclusões}
% Apresentar sucintamente suas constatações sobre o processo de aprendizado e
% desenvolvimento do app.
Um aprendizado que tivemos com esse EP é que certamente o desenvolvimento de
um programa para dispositivos móveis demanda mais preocupações do que uma
aplicação para notebooks ou desktops, por exemplo. Mesmo tratando-se de um
app bastante simples, o fato de precisarmos guardar informações para que o
usuário as possa acessar sem necessidade de conexão o tempo todo é o que
representa a maior parte da complexidade desse sistema tão simples.

Outro desafio adicionado pelo fato de nosso alvo serem os dispositivos móveis
é que a conformidade de comportamento é menos garantida do que suspeitávamos.
Mesmo num ambiente em que tudo parecia extremamente multiplataforma e
``seguro'' (HTML5 + JavaScript + CSS3, apenas com o PhoneGap fazendo a
intermediação necessária para cada plataforma), chegamos a observar divergência
de comportamento entre dois aparelhos com Android -- em certo ponto do
desenvolvimento, um deles apresentava
um erro que apareceu tão misteriosamente quanto depois sumiu, sem que
descobríssemos o motivo.

Por fim, vale ressaltar que os membros deste grupo possuíam pouco ou nenhum
conhecimento sobre JavaScript, de modo que o aprendizado acabou indo um pouco
além do escopo da disciplina de Computação Móvel.

\end{document}
